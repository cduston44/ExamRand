% Exam Template for UMTYMP and Math Department courses
%
% Using Philip Hirschhorn's exam.cls: http://www-math.mit.edu/~psh/#ExamCls
%
% run pdflatex on a finished exam at least three times to do the grading table on front page.
%
%%%%%%%%%%%%%%%%%%%%%%%%%%%%%%%%%%%%%%%%%%%%%%%%%%%%%%%%%%%%%%%%%%%%%%%%%%%%%%%%%%%%%%%%%%%%%%

% These lines can probably stay unchanged, although you can remove the last
% two packages if you're not making pictures with tikz.
\documentclass[11pt]{exam}
\usepackage[normalem]{ulem}
\usepackage{enumerate}
\RequirePackage{amssymb, amsfonts, amsmath, latexsym, verbatim, xspace, setspace}
\usepackage{graphicx}
%\RequirePackage{tikz, pgflibraryplotmarks}

%%%%%%%%%%%%%%%%%%%%%%%%%%%%%%%%%%%%%%%%%%
%
% Notice that this exam was remote! Don't
% use this for future templates!
%
%%%%%%%%%%%%%%%%%%%%%%%%%%%%%%%%%%%%%%%%%%

% By default LaTeX uses large margins.  This doesn't work well on exams; problems
% end up in the "middle" of the page, reducing the amount of space for students
% to work on them.
\usepackage[margin=1in]{geometry}
\usepackage{hyperref}


% Here's where you edit the Class, Exam, Date, etc.
\newcommand{\class}{Class}
\newcommand{\term}{Term}
\newcommand{\examnum}{Exam Number}
\newcommand{\examdate}{Date}
\newcommand{\timelimit}{Time Limit}
%\newcommand{\p}[1]

% For an exam, single spacing is most appropriate
\singlespacing
% \onehalfspacing
% \doublespacing

% For an exam, we generally want to turn off paragraph indentation
\parindent 0ex

% trying grey boxes
\usepackage{color}
\definecolor{lightgray}{gray}{0.75}

\newcommand\greybox[1]{%
  \vskip\baselineskip%
  \par\noindent\colorbox{lightgray}{%
    \begin{minipage}{0.8\textwidth}\centering #1 \end{minipage}%
  }%
  \vskip\baselineskip%
}

\begin{document} 

% These commands set up the running header on the top of the exam pages
\pagestyle{head}
\firstpageheader{}{}{}
\runningheader{\class}{\examnum\ - Page \thepage\ of \numpages}{\examdate}
\runningheadrule

\begin{flushright}
\begin{tabular}{p{2.8in} r l}
\textbf{\class} & \textbf{Name (Print):} & \makebox[2in]{\hrulefill}\\
\textbf{\term} &&\\
\textbf{\examnum} &&\\
\textbf{\examdate} &&\\
\end{tabular}\\
\end{flushright}
\rule[1ex]{\textwidth}{.1pt}

\begin{minipage}{0.5\textwidth}
  \begin{center}
    \uline{Vectors}
  \end{center}
  \[\vec{r}=r_x\hat{x}+r_y\hat{y}\]
  \[\vec{u}\cdot\vec{v}=uv\cos\theta=u_xv_x+u_yv_y\]
  \[\vec{u}\times\vec{v}=(u_yv_z-u_zv_y)\hat{x}-(u_xv_z-u_zv_x)\hat{y}+(u_xv_y-u_yv_x)\hat{z}\]
  \[|\vec{u}\times \vec{v}|=uv\sin\phi\]
\begin{center}
    \uline{Basic Definitions}
\end{center}
\[\vec{v}_{ave}=\frac{\Delta \vec{r}}{\Delta t},\qquad \vec{v}=\frac{d\vec{r}}{dt}\]
\[\vec{a}_{ave}=\frac{\Delta \vec{v}}{\Delta t},\qquad \vec{a}=\frac{d\vec{v}}{dt}\]
\[\omega_{ave}=\frac{\Delta \theta}{\Delta t},\qquad \omega=\frac{d\theta}{dt}.\]
\[v=r\omega,\qquad v_{cm}=R\omega\]
\begin{center}\uline{Energy and Interactions}\end{center}
\[K_{CM}=\frac{1}{2}mv^2,\qquad E=K+U\]
\[K_{rot}=\frac{1}{2}I\omega^2\]
\[\Delta E=0\qquad\qquad\text{(isolated)}\]
\[U_g=mgz,\quad U_G=-G\frac{m_1m_2}{r}\tag*{(Gravity)}\]
\[U_s=\frac{1}{2}k(r-r_0)^2\text{ or }U_s=\frac{1}{2}kx^2\tag*{(Spring)}\]
\[F_x=-\frac{dU}{dx}\tag*{(Force F)}\]
\[\Delta K=\vec{F}_A\cdot\Delta \vec{r}+\vec{F}_B\cdot \Delta \vec{r}+...\]
\[W=\vec{F}_A\cdot\Delta\vec{r}+\vec{F}_B\cdot\Delta\vec{r}+...\]
\end{minipage}
\begin{minipage}{0.5\textwidth}
\begin{center}\uline{Momentum and Impulse}\end{center}
\[\vec{p}=m\vec{v}\]
\[\Delta \vec{p}=\vec{I}=[\Delta\vec{p}]_A+[\Delta\vec{p}]_B+...\]
\[\Delta \vec{p}=0\qquad\qquad\quad~~\text{(isolated)}\]
\[\vec{r}_{CM}=\frac{m_1\vec{r}_1+m_2\vec{r}_2+...+m_N\vec{r}_N}{m_1+m_2+...+m_n}\]
\[\vec{L}=\vec{r}\times \vec{p},~~~ \text{ or   }~~~ \vec{L}=I\vec{\omega}\]
\[\Delta \vec{L}=0\qquad\qquad\quad~~\text{(isolated)}\]
\begin{center}\uline{Forces and Torques}\end{center}
\[\vec{F}_A=\frac{[d\vec{p}]_A}{dt},\qquad \vec{F}_g=m\vec{g}.\]
\[\vec{\tau}_A=\frac{[d\vec{L}]_A}{dt},\qquad \vec{\tau}=\vec{r}\times \vec{F}\]
\end{minipage}
\newpage



Instructions:\\

\begin{minipage}[t]{3.7in}
\vspace{0pt}
\begin{itemize}
\item Do this.

\item Don't do that.
  
\item \textit{Show all your work!} 



\item \textit{Solution Methods:} I will state in the problem if I want you to use a specific solution method - \textit{i.e.} kinematics or energy conservation. If you solve a problem correctly without using the requested method, you will receive partial credit but \textit{not} full credit. But using the alternate method to \textit{check} your answers is highly encouraged!

\item Please put \fbox{a box} around your final answer so I can easily see it. Units \textit{must} be included in final answers. I don't care about sig figs unless the problem specifically states that I do.
\end{itemize}
\end{minipage}
\hfill
\begin{minipage}[t]{2.3in}
\vspace{0pt}
%\cellwidth{3em}
\gradetablestretch{2}
\vqword{Problem}
\addpoints % required here by exam.cls, even though questions haven't started yet.	
\gradetable[v]%[pages]  % Use [pages] to have grading table by page instead of question

\end{minipage}

\newpage % End of cover pages

%%%%%%%%%%%%%%%%%%%%%%%%%%%%%%%%%%%%%%%%%%%%%%%%%%%%%%%%%%%%%%%%%%%%%%%%%%%%%%%%%%%%%
%
% See http://www-math.mit.edu/~psh/#ExamCls for full documentation, but the questions
% below give an idea of how to write questions [with parts] and have the points
% tracked automatically on the cover page.
%
%
%%%%%%%%%%%%%%%%%%%%%%%%%%%%%%%%%%%%%%%%%%%%%%%%%%%%%%%%%%%%%%%%%%%%%%%%%%%%%%%%%%%%%

\begin{questions}
\addpoints
\question[16] \textbf{Quick Calculations}
\noaddpoints
\begin{parts}
  \part[4] A bird with a mass \Pa kg is traveling with a speed \Pb m/s. How much kinetic energy does it have?
  
\vfill
\part[4] Another problem!
\vfill



\end{parts}
\newpage
\begin{figure}\begin{center}
    %\includegraphics[scale=0.25]{Rocketexplosion_clean}
    \end{center}\end{figure}
\addpoints
\question[12] A spaceship traveling at a speed of \Pc m/s crashes into a stationary asteroid of radius \Pd km. Does anyone care?


  \vfill
  \newpage
  


\end{questions}
\end{document}
